%`preamble' precedes the main text
\documentclass[12pt, a4paper]{article}

%some important packages
\usepackage[T1]{fontenc}					%package to set font encoding
\usepackage[latin1]{inputenc}				%package to set input encoding

%\usepackage[ngerman]{babel} 				%option for paper in german
\usepackage{graphicx} 						%package to include graphics
\usepackage{setspace} 						%package to set space between lines
\usepackage{lscape} 							%package to rotate pages
\usepackage{lipsum} 							%package for random text, delete \lipsum[1]-command when you write your own text
\usepackage{booktabs} 						%package for nice tables
\usepackage{amsmath,amsfonts,amssymb} 	%packages for LaTeX that provides various features to facilitate writing math 									
												%formulas and to improve the typographical quality of their output 
\usepackage{array}							%package for extending array and tabular environments
\usepackage{textcomp}						%package support of the text companion fonts
\usepackage{exscale} 							%package for nice summation sign 
\usepackage{tabularx} 						%package for column width and linebreaks in table cells
\usepackage{natbib}							%package for bibliography
\usepackage{ragged2e} 						%package provides less extreme raggedness than the standard LaTeX commands \flushleft and \flushright
\usepackage{multirow}           					%package to combine cells in tables
\usepackage{eurosym} 						%package to get eurosymbol per \euro
\usepackage{verbatim}						% package for verbatim evironment
\usepackage{float}							%package to define position of tables and figures

%adjustment of page parameters
\usepackage[left=3cm, right=2cm, top=2.5cm, bottom=2.5cm]{geometry}
\renewcommand{\baselinestretch}{1.5} 		%line spacing is set to 1.5
\setlength{\footnotesep}{10.0pt}				%space between footnote and text


%title
\title{{\Large Do voting systems affect segregation\footnote{This paper was written as a term paper in the course XYZ at the Chair of Economics VI: Empirical Economics, Prof. Dr. Mario Larch, University of Bayreuth.}}\\
\vspace{3mm}{\Large Term Paper, Software Engineering for Economists}%options: BA Thesis, MA Thesis etc.
}

%author
\author{Julia Baumann\footnote{4th semester, BA Economics. Student No.: 2323235. Address: Joseph-Schumpeter-Allee 42, 95444 Bayreuth. Tel.: +31 415 926 535 897. Email: market.frictions@uni-bayreuth.de.}\\ Tadas Gedminas\\Axel Purwin}
\date{Autumn 2017}
%beginning of the main text
\begin{document}

\maketitle \thispagestyle{empty}

%abstract
\begin{abstract}

%temporary set single space
\setlength{\baselineskip}{10.5pt} 

\vspace{0.5cm} 
\noindent You should put your abstract here. 
\vspace{0.5cm} 

{\normalsize \noindent \emph{JEL-Codes}:  Code1, Code2, Code3.\\ % for more information on JEL codes, consult: http://www.aeaweb.org/jel/jel_class_system.php
\emph{Keywords}: Keyword1, Keyword2, Keyword3.} % put some keywords that describe your work here
\end{abstract}

\newpage

%you can comment the following three lines if you write a term paper
\thispagestyle{empty}
\tableofcontents
\newpage

\newpage
\thispagestyle{empty}
\listoftables


\newpage
\thispagestyle{empty}
\listoffigures


\newpage


\setcounter{page}{1}%set page number to one

%*******************************************************************************************************************************************************************
%*******************************************************************************************************************************************************************
%*******************************************************************************************************************************************************************
%*******************************************************************************************************************************************************************
%main content

\section{\label{sec_intro}Introduction}

%*******************************************************************************************************************************************************************
%*******************************************************************************************************************************************************************

\section{The Model}\label{sec_model}



%*******************************************************************************************************************************************************************
%*******************************************************************************************************************************************************************

\section{\label{sec_strat}Estimation Strategy}


%*******************************************************************************************************************************************************************
%*******************************************************************************************************************************************************************

\section{\label{sec_data}Data}


%*******************************************************************************************************************************************************************
%*******************************************************************************************************************************************************************

\section{\label{sec_res}Results}







%*******************************************************************************************************************************************************************
%*******************************************************************************************************************************************************************

\section{\label{sec_conc}Conclusion}


%*******************************************************************************************************************************************************************
%*******************************************************************************************************************************************************************

%reference section with econometrica style. there are two options. the first is to use a bibtex-file. this can be very helpful in a paper with many reference. the second is to write down the referenes by hand. 

\newpage

%Version 1
%\bibliographystyle{econometrica}
%\bibliography{bibliography}

%Version 2
\section*{\label{sec_ref}References}
ANGRIST, J. D., AND J. PISCHKE (2009): \textit{Mostly Harmless Econometrics: An Empiricist's Companion}. Princeton University Press.
    
\noindent KENNEDY, P. (2005): ``Oh No! I Got the Wrong Sign! What Should I Do?,'' \textit{Journal of Economic Education}, 36, 77--92.
    
\noindent LEAMER, E. E. (1975): ```Explaining Your Results' as Access-Biased Memory,'' \textit{Journal of the American Statistical Association}, 70, 88--93.
    
\noindent LEAMER, E. E. (1983): ``Let's Take the Con Out of Econometrics,'' \textit{American Economic Review}, 73, 31--43.
    
\noindent SALA-I-MARTIN, X. X. (1997): ``I Just Ran Two Million Regressions,'' \textit{American Economic Review}, 87, 178--183.

\noindent WOOLDRIDGE, J. M. (2008): \textit{Introductory Econometrics: A Modern Approach}. South-Western Cengage Learning.

%*******************************************************************************************************************************************************************
%*******************************************************************************************************************************************************************



\end{document}
%the  end 